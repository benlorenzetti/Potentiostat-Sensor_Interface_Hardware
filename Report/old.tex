%%%%%%%%%%%%%%%%%%%%%%%%%%%%%%%%%%%%%%%%%
% Arsclassica Article
% LaTeX Template
% Version 1.1 (10/6/14)
%
% This template has been downloaded from:
% http://www.LaTeXTemplates.com
%
% Original author:
% Lorenzo Pantieri (http://www.lorenzopantieri.net) with extensive modifications by:
% Vel (vel@latextemplates.com)
%
% License:
% CC BY-NC-SA 3.0 (http://creativecommons.org/licenses/by-nc-sa/3.0/)
%
%%%%%%%%%%%%%%%%%%%%%%%%%%%%%%%%%%%%%%%%%

%----------------------------------------------------------------------------------------
%	PACKAGES AND OTHER DOCUMENT CONFIGURATIONS
%----------------------------------------------------------------------------------------

\documentclass[
10pt, % Main document font size
letterpaper, % Paper type, use 'letterpaper' for US Letter paper
oneside, % One page layout (no page indentation)
%twoside, % Two page layout (page indentation for binding and different headers)
headinclude,footinclude, % Extra spacing for the header and footer
BCOR5mm, % Binding correction
]{scrartcl}

\input{structure.tex} % Include the structure.tex file which specified the document structure and layout

\hyphenation{Fortran hy-phen-ation} % Specify custom hyphenation points in words with dashes where you would like hyphenation to occur, or alternatively, don't put any dashes in a word to stop hyphenation altogether

%----------------------------------------------------------------------------------------
%	TITLE AND AUTHOR(S)
%----------------------------------------------------------------------------------------

\title{\normalfont\spacedallcaps{Potentiostat--Sensor Interface Hardware}} % The article title

\author{\spacedlowsmallcaps{Ben Lorenzetti \& Wenjing Kang \& Ian Papautsky}} % The article author(s) - author affiliations need to be specified in the AUTHOR AFFILIATIONS block

\date{} % An optional date to appear under the author(s)

%----------------------------------------------------------------------------------------

\begin{document}

%----------------------------------------------------------------------------------------
%	HEADERS
%----------------------------------------------------------------------------------------

\renewcommand{\sectionmark}[1]{\markright{\spacedlowsmallcaps{#1}}} % The header for all pages (oneside) or for even pages (twoside)
%\renewcommand{\subsectionmark}[1]{\markright{\thesubsection~#1}} % Uncomment when using the twoside option - this modifies the header on odd pages
\lehead{\mbox{\llap{\small\thepage\kern1em\color{halfgray} \vline}\color{halfgray}\hspace{0.5em}\rightmark\hfil}} % The header style

\pagestyle{scrheadings} % Enable the headers specified in this block

%----------------------------------------------------------------------------------------
%	TABLE OF CONTENTS & LISTS OF FIGURES AND TABLES
%----------------------------------------------------------------------------------------

\maketitle % Print the title/author/date block

\setcounter{tocdepth}{2} % Set the depth of the table of contents to show sections and subsections only

\tableofcontents % Print the table of contents

%\listoffigures % Print the list of figures

%\listoftables % Print the list of tables

%----------------------------------------------------------------------------------------
%	ABSTRACT
%----------------------------------------------------------------------------------------

\section*{Abstract} % This section will not appear in the table of contents due to the star (\section*)

To best demonstrate the advantages of point-of-care electrochemical sensors,
a portable potentiostat with small form factor and simple user interface is needed.
To satisfy this need, the EmStat and Multiplexer OEM boards were purchased from PalmSens Compact Electrochemical Interfaces,
and a breakout board with protective packaging was developed to provide a clean user interface.

%----------------------------------------------------------------------------------------
%	AUTHOR AFFILIATIONS
%----------------------------------------------------------------------------------------

%{\let\thefootnote\relax\footnotetext{* \textit{Department of Biology, University of Examples, London, United Kingdom}}}

%{\let\thefootnote\relax\footnotetext{\textsuperscript{1} \textit{Department of Chemistry, University of Examples, London, United Kingdom}}}

%----------------------------------------------------------------------------------------

\newpage % Start the article content on the second page, remove this if you have a longer abstract that goes onto the second page

%----------------------------------------------------------------------------------------
%	INTRODUCTION
%----------------------------------------------------------------------------------------

\section{Introduction}

BioMicroSystems Lab is developing point-of-care electrochemical sensors for measuring heavy metal concentrations in environmental samples,
such as manganese and lead in lake water \cite{bms-article}.
Three electrodes are submerged in the sample and a potential is applied between two electrodes.
A different pair of two electrodes is used to measure current.
At some voltage level, an oxidation-reduction reaction between the metal of interest and the electrode surface becomes favorable.
As the metal changes oxidation state, charge is emitted so measuring and integrating the current gives the quantity of metal reactant in the sample.

In a three electrode system, one electrode must be common to both the voltage pair and the current pair,
and this is termed the counter electrode.
For the other two electrodes (reference and working) to accurately separate voltage and current,
they must be virtually grounded by amplifier electronics.
A device which drives the virtual ground, sets a potential, and measures resulting current is called a potentiostat.

The possible advantages of the point-of-care electrodes being developed by BMS Lab include
low cost, portability, measurement speed, and simplicity for non-technical operators.
To realize these advantages, a small, portable potentiostat is needed with a simple user interface.
 
%----------------------------------------------------------------------------------------
%	METHODS
%----------------------------------------------------------------------------------------

\section{Methods}

%Reference to Figure~\vref{fig:gallery}. % The \vref command specifies the location of the reference

%\begin{figure}[tb]
%\centering 
%\includegraphics[width=0.5\columnwidth]{GalleriaStampe} 
%\caption[An example of a floating figure]{An example of a floating figure (a reproduction from the \emph{Gallery of prints}, M.~Escher,\index{Escher, M.~C.} from \url{http://www.mcescher.com/}).} % The text in the square bracket is the caption for the list of figures while the text in the curly brackets is the figure caption
%\label{fig:gallery} 
%\end{figure}

%\lipsum[5] % Dummy text

%\begin{enumerate}[noitemsep] % [noitemsep] removes whitespace between the items for a compact look
%\item First item in a list
%\item Second item in a list
%\item Third item in a list
%\end{enumerate}

An OEM potentiostat board measuring only 2" x 1.35" was purchased from PalmSens, called EmStat.
Also from PalmSens, a Multiplexer was purchased that allows EmStat to switch between multiple sensors.
See figure \ref{emstat}.
However, before these could be used in the field,
a breakout board for connecting the Multiplexer to the BMS Lab sensors and protective packaging was needed.

\begin{figure}[tb]
\centering
\subfloat[EmStat Potentiostat \cite{palmsens-website}.]{\includegraphics[width=0.45\columnwidth]{EmStat}}	\quad
\subfloat[Multiplexer Pinout Diagram \cite{mux-documentation}.]{\includegraphics[width=0.45\columnwidth]{MUX_Pinout_Diagram}}	\\
\subfloat[EmStat and Multiplexer Connected \cite{palmsens-website}.]{\includegraphics[width=0.45\columnwidth]{EmStat_with_Multiplexer}}	\quad
\subfloat[Multiplexer Pinout Schematic. \cite{mux-documentation}]{\includegraphics[width=0.45\columnwidth]{MUX_Pinout_Schematic}}
\caption{EmStat and Multiplexer from PalmSens.}
\label{emstat}
\end{figure}

%----------------------------------------------------------------------------------------
%	BREAKOUT BOARD
%----------------------------------------------------------------------------------------

\section{Breakout Board}

%------------------------------------------------

\subsection{Multiplexer and Sensor Interfaces}

%------------------------------------------------

\subsection{Physical Constraints}

%------------------------------------------------

\subsection{Schematic Diagram}

%------------------------------------------------

\subsection{Board Design}

%----------------------------------------------------------------------------------------
%	PACKAGING
%----------------------------------------------------------------------------------------

\section{Packaging}

%------------------------------------------------

\subsection{Initial Base and Lid Designs}

%------------------------------------------------

\subsection{Revision History}

%------------------------------------------------

\subsection{3D Printing and Tapping}

%----------------------------------------------------------------------------------------
%	BIBLIOGRAPHY
%----------------------------------------------------------------------------------------

\renewcommand{\refname}{\spacedlowsmallcaps{References}} % For modifying the bibliography heading

\bibliographystyle{unsrt}

\bibliography{sample.bib} % The file containing the bibliography

%----------------------------------------------------------------------------------------

\end{document}
